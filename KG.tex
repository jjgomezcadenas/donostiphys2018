\begin{frame}
\frametitle{The Klein Gordon equation}
The Dirac equation describes spin-1/2 particles such electrons and neutrinos. It emerges from Dirac's attempt to avoid the negative solutions in the equation of Klein-Gordon which is obtained when one quantizes the relativistic relation:
\[
E^2 = p^2 + m^2 \, \, {(\rm with~ c = 1)}
\]
through the quantum-mechanical recipe:
\[
E \rightarrow i \frac{\partial}{\partial_t}, \,\, \vv{\bm{P}} \rightarrow -i \vv{\bm{\nabla}}
\]
then we obtain the Klein Gordon equation


\begin{empheq}[box=\fbox]{align}
   (i \frac{\partial}{\partial_t})^2 \psi= [(-i \bar{\nabla})^2 + m^2] \psi \nonumber
\end{empheq}

%\[
%(i \frac{\partial}{\partial_t})^2 \psi= [(-i \bar{\nabla})^2 + m^2] \psi
%\]
%which is the KG equation. 

The wavefunction $\psi$~ is now a relativistic scalar and the space and time derivatives are both
second order. However, the initial values of $\psi$~  and
$\partial \psi$~  can be chosen freely, and as a result the probability density is no longer positive definite. 
This leaves open the possibility of negative probabilities.
\end{frame}
