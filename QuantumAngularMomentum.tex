\begin{frame}
\frametitle{Quantum angular momentum and space quantization}
 
Quantum wave mechanics provides, via the quantum mechanical version of 
$\vv{\bm{L}} = \vv{\bm{r}} \times \vv{\bm{p}}$
a quantum description of the orbital angular momentum of a particle, such as that associated with an electron moving in an orbit around an atomic nucleus. The general results found are that the magnitude of the angular momentum is limited to the values:
\[
L = \sqrt{l(l+1)}\hbar, \, \, \, l = 0,1,2,3
\]

The quantum theory of orbital angular momentum also states that any one vector component of 
$\vv{\bm{L}}$, say $L_z$~Lz  is restricted to the values:
\[
L = m_l\hbar, \, \, \, m_l = -l, -l+1, -l+2....l-1, l.
\]

This restriction on the possible values of $L_z$~ mean that the angular momentum vector can have only
certain orientations in space --a result known as ``space quantization''.

All this is built around the quantum mechanical version of $\vv{\bm{L}} = \vv{\bm{r}} \times \vv{\bm{p}}$, and so implicitly is concerned with the angular momentum of a particle moving through space. In this sense the quantization of angular momentum is a necessary consequence of wave mechanics. \alert{However, the quantization of $\vv{\bm{L}}$, does not explain or predict the existence of an intrinsic angular momentum
$\vv{\bm{S}}$}.
\end{frame}
