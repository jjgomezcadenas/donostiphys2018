\begin{frame}
\frametitle{The classical view of Spin}
\begin{columns}
 
\column{0.5\textwidth}
\includegraphics[scale=0.1]{electronMagnet.png}
The cartoon pictures the ``classical'' view of an atom (hydrogen), seen as a tiny magnet. The electron ``orbiting'' around the nucleus is said to have an (orbital) angular momentum defined by:
\[
\vv{\bm{L}} = \vv{\bm{r}} \times \vv{\bm{p}} 
\]

From a classical perspective, as an electron carries a charge, its orbital motion will result in a tiny current loop which will produce a dipolar magnetic field. The strength of this dipole field is measured by the magnetic moment $\mu$ which is related to the orbital angular momentum by:
\[
\vv{\bm{\mu_L}} = \frac{q}{2m}\vv{\bm{L}} 
\]
\column{0.5\textwidth}
\includegraphics[scale=0.1]{classicalElectronSpin.png}
The classical idea of spin follows directly from the above considerations. Spin is the angular momentum we associate with a rotating sphere. If the sphere possesses an electric charge, then the circulation of the charge around the axis of rotation will constitute a current and hence will give rise to a magnetic field. This field is a dipole field whose strength, for a uniformly charged sphere of total charge q, is:

\[
\vv{\bm{\mu_L}} = \frac{q}{2m}\vv{\bm{S}} 
\]
\end{columns}



\end{frame}