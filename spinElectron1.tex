\begin{frame}
\frametitle{What do we talk about when we talk about spin? NRQM and Pauli matrices}

In NRQM we define $\ket{\uparrow}$~to be the state with spin in the $+z$~direction and 
$\ket{\downarrow}$~to be the state with spin in the $-z$~direction.
 
\[
\ket{\uparrow} = \mqty(1\\0), \,\,\, \ket{\downarrow} = \mqty(0\\1)
\]

In the space spanned by $\ket{\uparrow}$~and $\ket{\downarrow}$~the angular momentum operator is represented by the Pauli matrices
\[
J_i =\frac{\sigma_i}{2}, \,\,\, (i=1,2,3, ~{\rm or}~x,y,z)
\]
which satisfy the commutation relation of angular momentum operators, 
$[J_i, J_k] = i \epsilon_{ijk} J_k$. The square of the angular momentum operator $\va{J^2}$ is:
\[
\va{J^2} =J_1^2 + J_2^2 + J_2^2 = j(j+1)
\]
But then: 
\[
\frac{\va{\sigma^2}}{2} =(\frac{\sigma_1}{2})^2 + (\frac{\sigma_2}{2})^2 + (\frac{\sigma_3}{2})^2 
= \frac{3}{4} = \frac{1}{2}(\frac{1}{2} + 1)
\]
$\va{\sigma}/2$~ is  a spin-1/2 representation of angular momentum.
\end{frame}