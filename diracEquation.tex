\begin{frame}
\frametitle{The Dirac equation }
using the linearized equation
\[
E  -\va{\alpha}\cdot \va{P} - \beta \cdot m = 0
\]
and substituting operators
\[
E \rightarrow i \frac{\partial}{\partial_t}, \,\, \va{P} \rightarrow -i \va{\nabla}
\]
One obtains the Dirac equation
\[
 i \frac{\partial}{\partial_t} \psi = [ \va{\alpha} (-i\va{\nabla}) + m] \psi
\]

Multiply now $\beta$~from the left and define  the gamma matrices:
\[
 \gamma^0 = \beta = \begin{pmatrix} 
I & 0 \\
0 & -I 
\end{pmatrix}  \,\,\, ,  \gamma^i = \beta \alpha_i = \begin{pmatrix} 
0 & \sigma_i \\
-\sigma_i & 0 
\end{pmatrix} 
\]
To obtain 
%
%\begin{empheq}[box=\fbox]{align}
%(E \gamma^0 -\va{p}\cdot \va{\gamma} -m ) \psi & = 0 \nonumber
%\end{empheq}
%\[
%[i(\gamma^0 \partial_0 + \gamma^i \partial_i) -m ] \psi  = 0
%\]
%or
 \begin{empheq}[box=\fbox]{align}
(i \gamma^\mu \partial_\mu -m ) \psi & = 0 \nonumber
\end{empheq}
\end{frame}

%\begin{frame}
%Alternatively, in terms of the energy and momentum operators:
%\begin{empheq}[box=\fbox]{align}
%(E \gamma^0 -\va{p}\cdot \va{\gamma} -m ) \psi & = 0 \nonumber
%\end{empheq}
%
%In this equation $\psi$~is the Dirac bispinor: 
%\[
%\psi = \mqty(\psi_1 \\ \psi_2 \\ \psi_3 \\ \psi_4) =  \mqty(\phi \\ \chi); \,\,\,
%\phi =  \mqty(\phi_1 \\ \phi_2), \,\,\, \chi =  \mqty(\chi_1 \\ \chi_2)
%\]
%The two spinors $\phi$~and $\chi$~represent the particle and the antiparticle; the two components of each of them represent the two states of the third component of the spin, $s_z = +1/2$~and
%$s_z = -1/2$.
%
%The four $\gamma$~matrices, found before, are not unique. Any set that satisfy the anticommutation relations (Clifford algebra) can be used:
%
%\begin{empheq}[box=\fbox]{align}
%\{\gamma^\mu, \gamma^\nu\} = 2 g^{\mu\nu} \nonumber
%\end{empheq}
%\end{frame}
