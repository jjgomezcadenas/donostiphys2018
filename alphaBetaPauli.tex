\begin{frame}
\frametitle{Constructing $\bf{\alpha}$~and $\bf{\beta}$~using Pauli matrices }
%\begin{block}{}
%Since the $\alpha, \beta$~do not commute, they cannot be numbers. \alert{They need to be matrices}. In fact, they are traceless hermitian ($A^\dagger = A$) matrices or rank greater or equal than four. 
%\end{block}

They $\alpha, \beta$ can be constructed in terms of the  Pauli matrices:
\[
\alpha_i = 
\begin{pmatrix} 
0 & \sigma_i \\
\sigma_i & 0 
\end{pmatrix} \, \, ,
\beta = 
\begin{pmatrix} 
I & 0 \\
0 & -I 
\end{pmatrix} 
\]

where $I$~is the $2 \times 2$~ identity matrix, and the Pauli matrices are:
\[
\sigma_1 = 
\begin{pmatrix} 
0 & 1 \\
1& 0 
\end{pmatrix} \, \, ,
\sigma_2 = 
\begin{pmatrix} 
0 & -i \\
i & 0 \\ 
\end{pmatrix} \, \, ,
\sigma_3 = 
\begin{pmatrix} 
1 & 0 \\
0 & -1 \\ 
\end{pmatrix}
\]

Pauli matrices exhibit clearly the properties of being hermitian and traceless,
$\sigma_i^\dagger = \sigma_i$, $Tr \sigma_i = 0$, and $\sigma_i^2 = I$. They satisfy the commutation relations: 
\begin{empheq}[box=\fbox]{align}
 %\begin{empheq}[box=\tcbhighmath]{align}
\{\sigma_i, \sigma_j\}&=2 \delta_{ij} \, \, ( i,j,k = 1, 2, 3)\\
[\sigma_i, \sigma_j] & = 2 i \epsilon_{ijk} \sigma_k \nonumber
\end{empheq}
\end{frame}
