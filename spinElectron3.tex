\begin{frame}
\frametitle{What do we talk about when we talk about spin? Spin in the Dirac equation}

We can represent rotations in spinor space through the corresponding representation of the Lorentz group: 
 \[
 U(\va{\theta}) = e ^{-i \theta_i\frac{\Sigma_i}{2}}
 \]
where $\Sigma_i$~can be constructed in terms of the Pauli matrices:
 \[
 \Sigma_i = \mqty(\sigma_i & 0\\ 0 & \sigma_i)
 \]
We can now write the Dirac equation in terms of the Hamiltonian: 
\[
i \pdv{t} \psi = H \psi, \,\,\, H = \va{\alpha} \cdot \va{p} + \beta m
\]
%\end{frame}
and enquire if the angular momentum L commutes with H (ergo a constant of motion). Write:
\begin{eqnarray*}
L_i &= &(\va{x} \times \va{p})_i = \epsilon_{ijk} x^j p^k \\
H & = & \alpha_i p^i + \beta m = \gamma^0(\gamma^i p^i + m) 
\end{eqnarray*}
\end{frame}
\begin{frame}
then we find that: 
\[
[L_i, H] = [\epsilon_{ijk} x^j p^k,  \gamma^0(\gamma^l p^l + m] = i (\va{\alpha} \times \va{p})_i
\]
\[
[\Sigma_i, H] = [i \gamma^j \gamma^k,  \gamma^0(\gamma^l p^l + m] =-2i (\va{\alpha} \times \va{p})_i
\]
And thus, neither $L$, nor $\Sigma$~are constants of motion. Instead we find that:

\[
[\va{J}, H] = 0, \,\,\, \va{J} = \va{L} + \frac{\va{\Sigma}}{2}
\]

\begin{alertblock}{The Dirac equation describes the spin of the electron}
The angular momentum in the Dirac equation is conserved only when the spin is added to the orbital momentum. 
\end{alertblock} 

\end{frame}

\begin{frame}
\frametitle{Is Spin real?}
\begin{block}{}
\begin{itemize}
\item \alert{Yes}. We need spin to explain Stern-Gerlach. In fact we could define spin as the stuff that results in the Stern-Gerlach experiment behavior.
\item \alert{Yes}. Spin emerges out of the Dirac equation which describes the electron (and the neutrino) as spin 1/2 particle. 
\item Spin is a pure quantum effect without a classical counterpart. The rotating sphere used to pictorially describe it is just a (perhaps confusing) metaphor. 
\end{itemize}
\end{block}
\end{frame}