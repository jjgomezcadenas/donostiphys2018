\begin{frame}
\frametitle{The Stern Gerlach experiment: an attempt to test space quantization}
\begin{columns}
 
\column{0.5\textwidth}
\includegraphics[scale=0.2]{sternGerlach3.png}
In the Stern-Gerlach experiment (1922) a beam of silver atoms passed through an inhomogeneous magnetic field as shown in the sketch. The experiment was intended to test the space-quantization associated with the orbital angular momentum of atomic electrons, already described by the ``old quantum theory'' (Sommerfeld) which could be tested by making use of the fact that an orbiting electron will give rise to a magnetic moment proportional to the orbital angular momentum of the electron. 
\column{0.5\textwidth}
\includegraphics[scale=0.2]{sgSpots.jpg}
The result of the experiment is shown  (picture from the original paper). There is an intensity minimum in the center of the pattern, and the separation of the beam into two components is clearly seen. This result seemed to confirm Sommerfeld's quantum-theoretical prediction of spatial quantization. 
%Pauli, a notoriously skeptical physicist, remarked, ``Hopefully now even the incredulous Stern will be convinced about directional quantization'' (in a letter from Pauli to Gerlach 17 February 1922). 
\end{columns}



\end{frame}