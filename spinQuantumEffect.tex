\begin{frame}
\frametitle{Spin is a quantum effect}
 
The classical view of spin requires that the spinning sphere example has a non-zero radius. Classically a point particle can only have a spin angular momentum of zero and so it cannot have a magnetic moment. Thus, from the point-of-view of classical physics, elementary particles such as an electron, which are known to possess spin angular momentum, cannot be viewed as point objects -- they must be considered as tiny spinning spheres. But high energy scattering experiments show that \alert{elementary particles such as the electron behave very much as point particles}, with a radius smaller than $10^{-17}$~m. Yet they are found to possess magnetic moment and thus spin angular momentum of a magnitude equal (for the electron) to 
$\frac{\sqrt{3}}{2} \hbar$which requires the surface of the particle to be moving at a speed greater than that of light. 
 
\begin{alertblock}{The electron spin is a purely quantum effect}
The classical picture of an elementary particle as a tiny, rapidly rotating sphere is untenable.
\end{alertblock}
 \end{frame}
