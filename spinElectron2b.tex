\begin{frame}
\frametitle{Rotation matrix}

\includegraphics[scale=0.7]{rotation.png}
The z-axis can be rotated to the direction of $\va{s}$~  by first rotating around the y-axis by $\theta$~ and then around the original z-axis (not the rotated one) by $\phi$~, where $(\theta,\phi)$~ are the polar angles of the direction $\va{s}$. 
 \[
 \va{s} = (s_x, s_y, s_z) = (\sin\theta\cos\phi, \sin\theta\sin\phi, \cos\theta).
 \]

The rotation matrix is then: 
 \[
 R(\theta, \phi) = e^{-i\frac{\phi}{2}\sigma_z} e^{-i\frac{\theta}{2}\sigma_y}
 \]

 \end{frame}
 
 \begin{frame}
\frametitle{Eigenvectors of spin in an arbitrary direction}

 \begin{eqnarray*}
 \chi_{+} &= & u(\theta, \phi) \mqty(1\\0) = \frac{1}{\sqrt{2(1 + s_z)}}\mqty(1 + s_z\\ s_{+})\\
\chi_{-} &= & u(\theta, \phi) \mqty(0\\1) = \frac{1}{\sqrt{2(1 - s_z)}}\mqty( s_z - 1\\ s_{-})\\
\end{eqnarray*}
where
\[
s_{\pm} = s_x \pm i s_y
\]
The spin component in an arbitrary direction $\va{s}$~ is represented by the operator
$\va{s} \cdot \va{\sigma}$~(in units of the absolute value of the spin, which is 1/2)
 
 \begin{eqnarray*}
\va{s} \cdot \va{\sigma} &= &s_x \mqty(0 & 1\\ 1 & 0) + s_y \mqty(0 & -i\\ i & 0) + s_z \mqty(1 & 0\\ 0 & -1) \\
& = &  \mqty(s_z & s_x - i s_y\\ s_x + i s_y & -s_z) =
  \mqty(s_z & s_{-}\\ s_{+} &  -s_z) 
\end{eqnarray*}
 \end{frame}
 
 
  \begin{frame}
\frametitle{Projection operators}

Using the fact that $\va{s} \cdot \va{\sigma}$~has eigenvalues $\pm 1$, we can construct
projection operators which project out $\chi_{\pm}$ from any vector:

\[
P_{\pm} = \frac{1 \pm \va{s} \cdot \va{\sigma}}{2}
\]
Writing any vector $v$~as a linear combination of $\chi_{\pm}$, $v = c_{+}\chi_{+} + c_{-}\chi_{-}$, we can see that $P_{\pm}$~projects out $\chi_{\pm}$~out of $v$
\[
P_{+} v = c_{+} \chi_{+}, \,\,\, P_{-} v = c_{-} \chi_{-}
\]

 \end{frame}