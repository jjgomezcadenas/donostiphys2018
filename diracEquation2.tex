\begin{frame}
\frametitle{The Dirac equation }
The Dirac equation can also be written as: 

 \begin{empheq}[box=\tcbhighmath]{align}
(E \gamma^0 -\va{p}\cdot \va{\gamma} -m ) \psi & = 0
\end{empheq}

In this equation $\psi$~is the Dirac bispinor: 
\[
\psi = \mqty(\psi_1 \\ \psi_2 \\ \psi_3 \\ \psi_4) =  \mqty(\phi \\ \chi); \,\,\,
\phi =  \mqty(\phi_1 \\ \phi_2), \,\,\, \chi =  \mqty(\chi_1 \\ \chi_2)
\]
The two spinors $\phi$~and $\chi$~represent the particle and the antiparticle; the two components of each of them represent the two states of the third component of the spin, $s_z = +1/2$~and
$s_z = -1/2$.

The four $\gamma$~matrices, found before, are not unique. Any set that satisfy the anticommutation relations (Clifford algebra) can be used:

 \begin{empheq}[box=\tcbhighmath]{align}
\{\gamma^\mu, \gamma^\nu\} = 2 g^{\mu\nu}
\end{empheq}

\end{frame}

%\begin{frame}
%\frametitle{The Dirac representation of the $\gamma$~matrices} 
%
%\[
%\gamma^0 = \mqty(I & 0 \\ 0 & -I), \,\,\,  \gamma^i = \mqty(0 & \sigma_i \\ -\sigma_i & 0)
%\]
%where the elements are $2 \times 2$~matrices and the $\sigma$~are the Pauli matrices:
%\[
%\sigma^1 = \mqty(0 & 1 \\ 1 & 0), \,\,\,  \sigma^2 = \mqty(0 & -i \\ -i & 0), \,\,\,
%\sigma^3 = \mqty(1 & 0 \\ 0 & -1)
%\]
%To be explicit we can write the full expansion of the Dirac equation:
%\[
%\begin{split}
% i \left[
%\left( \begin{array}{cccc}
%1 & 0 & 0 &0 \\
%0 & 1 & 0 &0 \\
%0 & 0 & -1 &0 \\
%0 & 0 & 0 & -1 \\
%\end{array} \right ) \partial_0 +  
%\left( \begin{array}{cccc}
%0 & 0 & 0 &1 \\
%0 & 0 & 1 &0 \\
%0 & -1 & 0 &0 \\
%-1 & 0 & 0 &0 \\
%\end{array} \right ) \partial_1 + 
% \left( \begin{array}{cccc}
%0 & 0 & 0 &-i \\
%0 & 0 &  i &0 \\
%0 & i & 0 &0 \\
%-i & 0 & 0 &0 \\
%\end{array} \right ) \partial_2   +  \right. \\
%\left. \left( \begin{array}{cccc}
%0 & 0 & 1 &0 \\
%0 & 0 & 0 &-1 \\
%-1 & 0 & 0 &0 \\
%0 & 1 & 0 &0 \\
%\end{array} \right ) \partial_3 
%\right] \left( \begin{array}{c}
%\psi_1 \\
%\psi_2 \\
%\psi_3 \\
%\psi_4 \\
%\end{array} \right )  = m \left( \begin{array}{c}
%\psi_1 \\
%\psi_2 \\
%\psi_3 \\
%\psi_4 \\
%\end{array} \right )
%\end{split}
%\]
%\end{frame}
