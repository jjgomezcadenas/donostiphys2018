\begin{frame}
\frametitle{Stern Gerlach experiment: Spin is real}

There are 47 electrons surrounding the silver atom nucleus, of which 46 form a closed inner core of total angular momentum zero -- there is no orbital angular momentum, and the electrons with opposite spins pair off, so the total angular momentum is zero, and hence there is no magnetic moment due to the core. The one remaining electron also has zero orbital angular momentum, \alert{so the sole source of any magnetic moment is that due to the intrinsic spin of the electron.}
 
Thus, the experiment represents a direct measurement of one component of the spin of the electron, this component being determined by the direction of the magnetic field, here taken to be in the z direction.

There are two possible values for $S_z$~ corresponding to the two spots on the observation screen, implying that $s = 1/2$. The allowed values for the $z$~component of the spin are:
\[
S_z = \pm \frac{1}{2}\hbar
\]

Of course there is nothing special about the direction $z$. Any component of the spin of an electron will have only two values. If $\vv{\bm{n}}$ is a unit vector specifying some arbitrary direction then:
\[
\vv{\bm{S}} \cdot \vv{\bm{n}} = \pm \frac{1}{2}\hbar
\]
\end{frame}