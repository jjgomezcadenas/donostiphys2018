\begin{frame}
\frametitle{The explicit Dirac equation: alternatives to the $\gamma$~matrices}

\[
\begin{split}
 i \left[
\left( \begin{array}{cccc}
1 & 0 & 0 &0 \\
1 & 0 & 0 &0 \\
1 & 0 & 0 &0 \\
1 & 0 & 0 &0 \\
\end{array} \right ) \partial_0 +  
\left( \begin{array}{cccc}
1 & 0 & 0 &0 \\
1 & 0 & 0 &0 \\
1 & 0 & 0 &0 \\
1 & 0 & 0 &0 \\
\end{array} \right ) \partial_1 + 
 \left( \begin{array}{cccc}
1 & 0 & 0 &0 \\
1 & 0 & 0 &0 \\
1 & 0 & 0 &0 \\
1 & 0 & 0 &0 \\
\end{array} \right ) \partial_2   +  \right. \\
\left. \left( \begin{array}{cccc}
1 & 0 & 0 &0 \\
1 & 0 & 0 &0 \\
1 & 0 & 0 &0 \\
1 & 0 & 0 &0 \\
\end{array} \right ) \partial_3 
\right] \left( \begin{array}{c}
\psi_1 \\
\psi_2 \\
\psi_3 \\
\psi_4 \\
\end{array} \right )  = m \left( \begin{array}{c}
\psi_1 \\
\psi_2 \\
\psi_3 \\
\psi_4 \\
\end{array} \right )
\end{split}
\]
%\begin{alertblock}{Important theorem}
Searching for a linear equation, Dirac found a system of 4 equations, operating over a 4-dimensional waveform. Notice, however, that the $\gamma$~matrices are not the only explicit expression that satisfies the Clifford algebra $\{\gamma^\mu, \gamma^\nu\} = 2 g^{\mu\nu}$. There are other possible representations, such as the Weyl, and the Majorana representation. The former is specially well suited to describe massless neutrinos. The later was used by Majorana to propose the hypothesis that the neutrino could be its own antiparticle. 
%\end{alertblock}
\end{frame}
